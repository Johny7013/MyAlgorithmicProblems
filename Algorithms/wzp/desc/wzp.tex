\documentclass[zad]{soigstyl}

% opcjonalne (obowiazkowe) parametry
\pagestyle{fancy}
\konkurs{Konkurs}
\etap{etap 1}
\day{1}
\date{01.01.1970}
%\setlogo{oiglogo}%brak ustawienia wstawia domyslnie logo talentu
%\setstopka{stopka_bw}
\title{\mbox{Wielki Zjazd Plików}}
\id{wzp}
\RAM{64}

\begin{document}

\begin{tasktext}%
    \noindent
    W~Bajtocji odbywa~się Wielki Zjazd Plików. Pliki z~całej Bajtocji zebarały~się w~Bajtogrodzie, aby dyskutować na~bardzo ważne tematy. Jesteś pracownikiem punktu informacyjnego i~dysponujesz listą, na~której znajdują~się wielkości plików, które przybyły na~WZP. Co~chwilę przychodzą do Ciebie pliki z~pytaniem: \textit{Ile jest na~zebraniu plików większych ode mnie?}. Twoim zadaniem jest odpowiedzienie na~$t$ takich pytań.
	
    	\section{Wejście}
	W~pierwszym wierszu standardowego wejścia znajdują~się dwie liczby naturalne $n$~i~$t$~($1 \leqslant n, t \leqslant 10^5$), które oznaczają odpowiednio liczbę plików, które pojawiły się na~WZP i~liczbę zapytań skierowanych do~Ciebie przez pliki. W~drugim wierszu podane jest $n$ liczb $a_1, a_2,...a_n$ ($1 \leqslant a_i \leqslant 10^9$), które oznaczaja rozmiary plików w~bitach. Kolejne $t$~wierszy zawiera po jednej liczbie naturalnej, której wartość nie~przkracza $10^5$ - są~to~numery plików, które zadawały kolejne pytania. Pliki są~numerowane w~takiej kolejności jak~zostały podane na~wejściu.

	\section{Wyjście}
	Na~standardowe wyjście należy wypisać~$t$~wierszy. W $j$-tym wierszu powinna znaleźć~się odpowiedź dla $j$-tego zapytania.
	
	\oigprzyklady
\end{tasktext}






\begin{flushright}
\textit{Autor zadania: Jan Klinkosz}\\
\end{flushright}
\end{document}
