\documentclass[zad]{soigstyl}

% opcjonalne (obowiazkowe) parametry
\pagestyle{fancy}
\konkurs{Konkurs}
\etap{etap 1}
\day{1}
\date{01.01.1970}
%\setlogo{oiglogo}%brak ustawienia wstawia domyslnie logo talentu
%\setstopka{stopka_bw}
\title{\mbox{Zakład}}
\id{zak}
\RAM{64}

\begin{document}

\begin{tasktext}%
    \noindent
    Bajtek i~Bitek bardzo lubią matematykę. Pewnego dnia na~lekcji matematyki nauczyli~się wyznaczać promień okręgu wpisanego w~sześciokąt foremny. Byli tym strasznie zafascyniwani. Dodatkowo obydwaj~są miłośnikami hazardu i~postanowili zrobić zakład. Bitek założył~się z~Bajtkiem o~to, że jest on~w~stanie w~jeden dzień policzyć $n$ promieni sześciąkątów foremnych, kiedy ma~dane tylko i~wyłącznie długości ich boków. Niestety Bitek przeliczył~się i~już wie, że nie zdąży zrobić tego sam. Zwrócił~się do Ciebie o~pomoc w~rozwiązaniu tego problemu. Napisz program, który dla $n$~sześciokątów foremnych znając długości ich boków policzy długości promieni okregów wpisanych w~te figury foremne. Pamiętaj, że Bitek liczy na Ciebie. Dodatkowo nadmienił, że jeżeli mu pomożesz, to~podzieli się on~z~Tobą swoją wygraną.
	
    	\section{Wejście}
	W~pierwszym wierszu standardowego wejścia znajduje~się jedna liczba naturalna $n$~($1 \leqslant n \leqslant 10^5$), która oznacza liczbę sześciokątów formnych, których promienie okręgów wpisanych należy wyznaczyć. W~kolejnych $n$~wierszach znajdują~się długości boków kolejnych sześciokątów wyrażone w~liczbach naturalnych z zakresu do $10^{18}$.

	\section{Wyjście}
	Na~standardowe wyjście należy wypisać $n$~wierszy. W~$i$-tym wierszu powinna znaleźć się~odpowiedź dla~$i$-tego zapytania wypisana z~trzema liczbami po przeciku zaokrąglona do~części tysięcznych.
	
	\oigprzyklady
\end{tasktext}
\end{document}
