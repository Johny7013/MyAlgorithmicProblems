\documentclass[zad]{soigstyl}

% opcjonalne (obowiazkowe) parametry
\pagestyle{fancy}
\konkurs{Konkurs}
\etap{etap 1}
\day{1}
\date{01.01.1970}
%\setlogo{oiglogo}%brak ustawienia wstawia domyslnie logo talentu
%\setstopka{stopka_bw}
\title{\mbox{Inwersacje}}
\id{inwersacje}
\RAM{64}

\begin{document}

\begin{tasktext}%
    \noindent
     Inwersją nazywamy taką parę $i$ oraz~$j$, że~$i < j$ i~$a_i > a_j$. Twoim zadaniem jest znalezienie liczby inwersji występujących we~wszystkich niepóźniejszych leksykograficznie permutacjach w~stosunku do~permutacji podanej na~wejściu.
	
    	\section{Wejście}
	W~pierwszym wierszu standardowego wejścia zapisano jedną liczbę całkowitą~$n$ ($1\leqslant n \leqslant 10^5$), oznaczającą długość permutacji. W~następnym wierszu znajduje~się permutacja liczb od~$1$ do~$n$. 

	\section{Wyjście}
	W~pierwszym wierszu standardowego wyjścia powinna znaleźć~się jedna liczba całkowita, oznaczająca liczbę inwersji we~wszystkich niepóźniejszych leksykograficznie permutacjach długości $n$ modulo $10^9 + 9$.
	
	\oigprzyklady
\end{tasktext}
\end{document}
