\documentclass[zad]{soigstyl}

% opcjonalne (obowiazkowe) parametry
\pagestyle{fancy}
\konkurs{Konkurs}
\etap{etap 1}
\day{1}
\date{01.01.1970}
%\setlogo{oiglogo}%brak ustawienia wstawia domyslnie logo talentu
%\setstopka{stopka_bw}
\title{\mbox{Pierwsze Kwadraty}}
\id{pkw}
\RAM{64}

\begin{document}

\begin{tasktext}%
    \noindent
    Bajtek jest bardzo pilnym uczniem. Zawsze uczy~się na~sprawdziany i~odrabia zadania domowe. Jego ulubionym przedmiotem jest matematyka. Na~ostatnich lekcjach poznał liczby pierwsze oraz~podnoszenie do~kwardatu, aby~powtórzyć sobie do~sprawdzianu postanowił połączyć dwa tematy w~jedno zadanie. Oblicza, ile kwadratów liczb pierwszych znajduje~się w~różnych przedziałach. Nie wie jednak jak szybko zweryfikować swoje wyniki. Napisz program który mu~w~tym~pomoże.
	
    	\section{Wejście}
	W~pierwszym wierszu standardowego wejścia znajduje~się jedna liczba naturalna $n$ ($1 \leqslant n \leqslant 10^6$) oznaczająca ilość zapytań Bajtka. W~kolejnych $n$ wierszach znajdują się po~dwie liczby naturalne $a_i$ i~$b_i$  ($1 \leqslant a \leqslant b \leqslant 10^{12}$), które oznaczają przedział liczb o~jaki pyta~się Bajtek w~$i-tym$ zapytaniu.
Dodatkowo w $30\%$ testów $n \leqslant 1000$.

	\section{Wyjście}
	Wyjście powinno składać się z~$n$ wierszy. W~$i-tym$ wierszu powinna znaleźć się odpowiedź dla $i-tego$ zapytania Bajtka.
	
	\oigprzyklady
\end{tasktext}
\end{document}
