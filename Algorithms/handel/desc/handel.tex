\documentclass[zad]{soigstyl}

% opcjonalne (obowiazkowe) parametry
\pagestyle{fancy}
\konkurs{Konkurs}
\etap{etap 1}
\day{1}
\date{01.01.1970}
%\setlogo{oiglogo}%brak ustawienia wstawia domyslnie logo talentu
%\setstopka{stopka_bw}
\title{\mbox{Handel}}
\id{handel}
\RAM{64}

\begin{document}

\begin{tasktext}%
    \noindent
    Bajtocja jest pięknym krajem, w~którym znajduje~się $n$ miast połączonych $n-1$ drogami. Dodatkowo w~Bajtocji można przejechać między każdymi dwoma miastami na~dokładnie jeden sposób. Niestety króla Bajtocji trapi jedna kwestia, a~mianowicie~to, że~hadel w~królestwie nie rozwija~się zbyt prężnie. Postanowił on~jednak temu zaradzić. Wydał dekret na~mocy którego każde miasto w~Bajtocji z~którego wychodzą co najmniej trzy drogi musi wybudować u~siebie punkt handlowy. Teraz Król zastanawia~się, jaka jest odległość między dwoma najdalej od~siebie oddalonymi punktami handlowymi. Zlecił Tobie, nadwornemu geografowi, podanie mu~odpowiedzi na~to~pytanie. Możesz założyć, że~zawsze będą istniały przynajmniej dwa punkty handlowe.
	
    	\section{Wejście}
	W~pierwszym wierszu standardowego wejścia znajduje~się jedna liczba naturalna~$n$ ($6 \leqslant n \leqslant 10^6$), oznaczająca liczbę miast w~Bajtocji. W~kolejnych $n-1$ wierszach znajdują~się dwie liczby naturalne $a$ i~$b$ ($1 \leqslant a, b \leqslant n$), które oznaczają, że między miastami $a$ i~$b$ znajduje~się dwukierunkowa droga.\\
Możesz założyć, że w~$20\%$ testów zachodzi dodatowy warunek: $n \leqslant 5000$.

	\section{Wyjście}
	Na~standardowe wyjście należy wypisać jedną liczbę naturalną, odległość między dwoma najdalej od~siebie oddalonymi punktami handlowymi.
	
	\oigprzyklady
\end{tasktext}
\end{document}
