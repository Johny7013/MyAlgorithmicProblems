\documentclass[zad]{soigstyl}

% opcjonalne (obowiazkowe) parametry
\pagestyle{fancy}
\konkurs{Konkurs}
\etap{etap 1}
\day{1}
\date{01.01.1970}
%\setlogo{oiglogo}%brak ustawienia wstawia domyslnie logo talentu
%\setstopka{stopka_bw}
\title{\mbox{Pole dowolnego wielokąta}}
\id{pdw}
\RAM{64}

\begin{document}

\begin{tasktext}%
    \noindent
    Dysponując współrzędnymi kolejnych wierzchołków dowolnego wielokąta oblicz jego pole.
	
    	\section{Wejście}
	W~pierwszym wierszu standardowego wejścia zapisano liczbę naturalną $N$
 $(4 \leqslant N \leqslant 300)$ - liczbę wierzchołków, których oddzielone spacją współrzędne znajdują się w następujących $n$ liniach $(-30000 \leqslant X, Y \leqslant 30000)$. 

	\section{Wyjście}
	W~jedynym wierszu standardowego wyjścia wypisz obliczoną wartość pola wielokąta z dokładnością do $0.01$.
	
	\oigprzyklady
\end{tasktext}
\end{document}
