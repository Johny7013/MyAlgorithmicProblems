\documentclass[zad]{soigstyl}

% opcjonalne (obowiazkowe) parametry
\pagestyle{fancy}
\konkurs{Konkurs}
\etap{etap 1}
\day{1}
\date{01.01.1970}
%\setlogo{oiglogo}%brak ustawienia wstawia domyslnie logo talentu
%\setstopka{stopka_bw}
\title{\mbox{Kostka lodu}}
\id{kostka}
\RAM{64}

\begin{document}

\begin{tasktext}%
    \noindent
    W~Bajtocji jest wyjątkowo upalne lato. Marek chce umilić sobie przebywanie w~swoim prostopadłościennym pokoju, więc postanowił włożyć do niego kostkę lodu o~masie $n$. Marek chce dowiedzieć~się o~ile zmniejszy~się temperatura w~jego pokoju po stopieniu całego lodu(kiedy temperatura wody będzie wynosiła $0$). Przy rozwiązywaniu zadania przyjmij, że: gęstość powietrza wynosi $d = 1,2 \frac{kg}{m^3}$, ciepło właściwe powietrza przy stałym ciśnieniu wynosi $c_w = 1020 \frac{J}{kg*K}$, ciepło topnienia lodu wynosi $c_t = 333,7 \frac{kJ}{kg}$, powietrze wypełnia całą objętość pokoju Marka.
	
    	\section{Wejście}
	W~pierwszym wierszu standardowego wyjścia znajduje~się jedna liczba całkowita $n~(1 \leqslant n \leqslant 10^3)$, oznaczająca masę kostki lodu w kilogramach. W~drugim wierszu znajdują się trzy liczby całkowite $a$, $b$ i~$c~(1\leqslant a, b, c \leqslant 10^3)$, oznaczające wymiary pokoju Marka w~centymetrach.

	\section{Wyjście}
	W~pierwszym wierszu standardowego wyjścia powinna znaleźć~się szukana wartość, ilość stopni w~skali Kelvina o~jaką zmniejszy~się temperatura w~pokoju Marka z~dokładnością do $0,001$ stopnia.
	
	\oigprzyklady
\end{tasktext}
\end{document}
