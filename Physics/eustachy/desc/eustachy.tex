\documentclass[zad]{soigstyl}

% opcjonalne (obowiazkowe) parametry
\pagestyle{fancy}
\konkurs{Konkurs}
\etap{etap 1}
\day{1}
\date{01.01.1970}
%\setlogo{oiglogo}%brak ustawienia wstawia domyslnie logo talentu
%\setstopka{stopka_bw}
\title{\mbox{Problem Eustachego}}
\id{eustachy}
\RAM{64}

\begin{document}

\begin{tasktext}%
    \noindent
    Eustachy zawsze fascynował~się grawitacją Ziemi.  Wiedział, że jej siła jest bardzo duża, bo nasza planeta ma masę $6 \cdot 10^{24}$ kilogramów, promień $6,4 \cdot 10^6$ metrów i~jest idealną kulą. Często pytał kolegów ile wynosi stała grawitacji $G$, ale oni niestety mieli go jak zawsze w~nosie, więc w~końcu sprawdził to w~internecie. Tam znalazł, że wynosi ona $6,67 \cdot 10^{-11}$. Ulubiona zabawą Eustachego było rzucanie piłką w~przód i~obserwowanie jej toru lotu w~szczytnym celu. Niestety ostatnio złamał sobie rękę, dlatego poprosił ciebie, żebyś za niego rzucił piłką i~sprawdził czy spadnie na ziemię! W~zadaniu pomiń opory powietrza. 
	
    	\section{Wejście}
	W~pierwszym i~jedynym wierszu wejścia otrzymujemy jedną liczbę całkowitą $v~(1 \leqslant v \leqslant 10^4)$, która oznacza szybkość z jaką rzucasz piłkę.

	\section{Wyjście}
	Na wyjście należy wypisać \texttt{TAK}, jeśli piłka spadnie na ziemię lub \texttt{NIE} w~przeciwnym przypadku.
	
	\oigprzyklady
\end{tasktext}
\end{document}
