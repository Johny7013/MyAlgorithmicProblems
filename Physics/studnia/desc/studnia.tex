\documentclass[zad]{soigstyl}

% opcjonalne (obowiazkowe) parametry
\pagestyle{fancy}
\konkurs{Konkurs}
\etap{etap 1}
\day{1}
\date{01.01.1970}
%\setlogo{oiglogo}%brak ustawienia wstawia domyslnie logo talentu
%\setstopka{stopka_bw}
\title{\mbox{Studnia}}
\id{studnia}
\RAM{64}

\begin{document}

\begin{tasktext}%
    \noindent
    Mały Mikołaj bardzo lubi bawić~się na podwórku, a~szczególnie w~pobliżu studni, która~się na nim znajduje. Pewnego dnia, gdy wrzucał kamienie do owej studni, zastanowiła go jedna kwestia, a~mianowicie po jakim czasie od wypuszczenia z~ręki kamienia słyszy on plusk wody. Niestety Mikołaj jest mały i~nie jest w~stanie policzyć tego sam. Poprosił ciebie, abyś pomógł mu rozwiązać to zadanie.
	
    	\section{Wejście}
	Na wejściu otrzymujemy jedną liczbę całkowitą $h~(1 \leqslant h \leqslant 10^6)$, która oznacza wysokość w~metrach od ręki Mikołaja do poziomu wody w~studni. Zakładamy, że prędkość dźwięku w~powietrzu wynosi $v=340\frac{m}{s}$ i~głowa Mikołaja znajduje~się o~$0,5 m$ wyżej niż jego ręka podczas wypuszczania kamienia.
	
	\section{Wyjście}
	W~pierwszym i~jedynym wierszu wyjścia należy wypisać czas (w~sekundach) po jakim Mikołaj usłyszy plusk z~dokładnością do $0.001$.
	
	\oigprzyklady
\end{tasktext}
\end{document}
