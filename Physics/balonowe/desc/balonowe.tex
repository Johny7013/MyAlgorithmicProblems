\documentclass[zad]{soigstyl}

% opcjonalne (obowiazkowe) parametry
\pagestyle{fancy}
\konkurs{Konkurs}
\etap{etap 1}
\day{1}
\date{01.01.1970}
%\setlogo{oiglogo}%brak ustawienia wstawia domyslnie logo talentu
%\setstopka{stopka_bw}
\title{\mbox{Oddziały balonowe}}
\id{balonowe}
\RAM{64}

\begin{document}

\begin{tasktext}%
    \noindent
    Wojtek zawsze chciał zakosztować wesołego życia żołnierza. Pewnego dnia otrzymał od dziadka wehikuł czasu, dzięki któremu mógł przenieść się do czasów dwudziestolecia międzywojennego w~Polsce. Tak~się akurat złożyło, że Wojtek znalazł~się w~odziałach wojska balonowego II RP. Podczas ćwiczeń ciekawość Wojtka wzbudziła jedna kwestia, a~mianowicie po jakim czasie od spuszczenia bomby uderzy ona o~ziemię. Przy rozwiązywaniu zadania przyjmij, że: przyśpieszenie ziemskie wynosi $g = 9,81 \frac{m}{s^2}$, balon porusza~się ruchem jednostajnym prostoliniowym w~górę oraz startuje z~powierzchni ziemi.
	
    	\section{Wejście}
	W~pierwszym i~jedynym wierszu standardowego wejścia znajdują~się dwie liczby całkowite $t$ i~$v~(1 \leqslant t, v \leqslant 10^5),$ które oznaczają odpowiednio czas(w sekundach) od rozpoczęcia ruchu balonu w~górę do spuszczenia bomby przez Wojtka i~wartość prędkości(w~$\frac{km}{h}$) z~jaką porusza się balon.

	\section{Wyjście}
	W~pierwszym wierszu standardowego wyjścia należy wypisać czas (w~sekundach) jaki upłynął od wypuszczenia bomby z~balonu do uderzenia jej o~podłoże z~dokładnością do $0,001$.
	
	\oigprzyklady
\end{tasktext}
\end{document}
