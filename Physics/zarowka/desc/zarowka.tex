\documentclass[zad]{soigstyl}

% opcjonalne (obowiazkowe) parametry
\pagestyle{fancy}
\konkurs{Konkurs}
\etap{etap 1}
\day{1}
\date{01.01.1970}
%\setlogo{oiglogo}%brak ustawienia wstawia domyslnie logo talentu
%\setstopka{stopka_bw}
\title{\mbox{Żarówka}}
\id{zarowka}
\RAM{64}

\begin{document}

\begin{tasktext}%
    \noindent
    Marek po długich perypetiach z~informatyką odkrył w~sobie jednak duszę fizyka i~postanowił podążyć tą drogą. Dzięki długiej i~żmudnej pracy, która jednak dla Marka była wielką przygodą, osiągnął on mistrzostwo w~dziedzinie fizyki kwantowej. Jego wielkie osiągnięcia naukowe spowodowały, że rząd Bajtocji postanowił zwerbować go do supertajnego projektu Bajhattan. Podczas pierwszych dni swojej nowej pracy Markowi zostało przydzielone proste zadanie, niestety nie z~jego ulubionego działu fizyki, a~mianowice: Marek ma policzyć jaka moc "odłoży się" na żarówce połączonej szeregowo z~dwoma opornikami i~jednym układem trzech oporników połączonych między sobą równolegle. Cały obwód podłączony jest do napięcia stałego $U$. Dla Marka jest to banalne zadanie i~bardzo nie chce mu~się go robić. Na~jego szczęście został mu również przydzielony asystent, którym jesteś ty. Marek postanowił wysłużyć~się tobą i~zlecił ci, abyś rozwiązał za niego to zadanie. 
	
    	\section{Wejście}
	Na wejściu otrzymujemy siedem liczb całkowitych $R_1$, $R_2$, $R_3$, $R_4$, $R_5$, $U$ i~$P~(1 \leqslant R_1, R_2, R_3, R_4, R_5, U, P \leqslant 10^4)$, które oznaczają odpowiednio: $R_1$ - opór pierwszego opornika podłączonego szeregowo w~omach, $R_2$ - opór drugiego opornika podłączonego szeregowo w~omach, $R_3$, $R_4$ i~$R_5$ - opory oporników połączonych ze sobą równolegle w~omach, $U$ - napięcie źródła prądu w~woltach, $P$ - moc nominalna żarówki w~watach (przy podanym na wejściu napięciu).

	\section{Wyjście}
	W~pierwszym i~jedynym wierszu wyjścia należy wypisać moc w~watach, która "odłoży~się" na żarówce. Twój błąd względny lub bezwzględny nie może przekraczać $10^{-3}$.
	
	\oigprzyklady
\end{tasktext}
\end{document}
