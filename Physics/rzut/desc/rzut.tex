\documentclass[zad]{soigstyl}

% opcjonalne (obowiazkowe) parametry
\pagestyle{fancy}
\konkurs{Konkurs}
\etap{etap 1}
\day{1}
\date{01.01.1970}
%\setlogo{oiglogo}%brak ustawienia wstawia domyslnie logo talentu
%\setstopka{stopka_bw}
\title{\mbox{Rzut Eustachego v2}}
\id{rzut}
\RAM{64}

\begin{document}

\begin{tasktext}%
    \noindent
    Eustachy ponownie postanowił pobawić~się swoją piłką. Tym razem nie nadaje jej prędkości o kierunku pionowym lecz tylko i~wyłącznie poziomym. Tak samo jak poprzednio rzuca piłkę o~masie~$m$ z~pewnej wysokości~$h$. Eustachy zastanawia się z jaką siłą skierowaną w~poziomie będzie działała ziemia na piłkę, która została rzucona z~prędkością~$v$. Przy rozwiązywaniu zadania pomiń wszystkie opory oprócz siły oporu ziemi i~przyjmij, że: stała grawitacji $G = 6,674 \times 10^{-11}\frac{m^3}{kg*s^2}$, promień ziemi wynosi $R = 6,378 \times 10^{6} m$, masa Ziemi wynosi $M=5,972 \times 10^{24} kg$, przyspieszenie ziemskie dla piłki jest stałe podczas całego ruchu i wynosi taką samą wartość jak wtedy, kiedy piłka znajduje sie w najwyższym punkcie, piłka po uderzeniu w ziemię porusza się dokładnie w tym samym kierunku jak zwrot prędkości w momencie uderzenia, a siła oporu ziemi ma przeciwny zwrot do wektora prędkości.
	
    	\section{Wejście}
	W pierwszym i~jedynym wierszu standardowego wejścia podane są cztery liczby całkowite $m, V, h, Fz$ ($10 \leqslant m \leqslant 15, 1 \leqslant V \leqslant 100000, 1 \leqslant h \leqslant 1000, 150 \leqslant F_z \leqslant 1000$), które oznaczają odpowiednio masę piłki w~$kg$, prędkość początkową piłki w~$\frac{km}{h}$, wysokość z~której Eustachy rzucił piłkę w metrach i~opór ziemi w~niutonach.

	\section{Wyjście}
	Na~standardowe wyjście należy wypisać siłę z jaką ziemia będzie działała na piłkę w kierunku poziomym, wyrażoną w niutonach z zaokrągleniem do $0,001$.
	
	\oigprzyklady
\end{tasktext}
\end{document}
