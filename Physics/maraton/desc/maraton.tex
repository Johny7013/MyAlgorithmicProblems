\documentclass[zad]{soigstyl}

% opcjonalne (obowiazkowe) parametry
\pagestyle{fancy}
\konkurs{Konkurs}
\etap{etap 1}
\day{1}
\date{01.01.1970}
%\setlogo{oiglogo}%brak ustawienia wstawia domyslnie logo talentu
%\setstopka{stopka_bw}
\title{\mbox{Maraton}}
\id{maraton}
\RAM{64}

\begin{document}

\begin{tasktext}%
    \noindent
    Piotr, jako nadzwyczajnie wysportowany człowiek, z~niemałą kondycją zdecydował~się wystartować w~pewnym maratonie. Organizatorzy owego maratonu zdecydowali, że co $\frac{1}{3}$ trasy będą rozmieszczone pomiary czasu, które zmierzą czas jak również wyliczą średnią prędkość na danym odcinku. Bohater naszej historii po ukończonym biegu, który oczywiście wygrał, chciał~się dowiedzieć z~jaką średnią wartością prędkości poruszał~się na całym odcinku maratonu. Jednakże organizatorzy nie przewidzieli takiej sytuacji, a~jako że nie potrafią tego wyliczyć i~Piotr jest bardzo uparty, wyznaczyli ciebie do pomocy.
	
    	\section{Wejście}
	W~pierwszym i~jedynym wierszu są podane trzy liczby całkowite oznaczające wartości prędkości Piotra na każdej $\frac{1}{3}$ długości trasy $(1 \leqslant v_1, v_2, v_3 \leqslant 10^6)$.

	\section{Wyjście}
	Na wyjściu powinna znaleźć~się średnia wartość prędkości Piotra na całej trasie.
	
	\oigprzyklady
\end{tasktext}
\end{document}
