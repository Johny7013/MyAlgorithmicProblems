\documentclass[zad]{soigstyl}

% opcjonalne (obowiazkowe) parametry
\pagestyle{fancy}
\konkurs{Konkurs}
\etap{etap 1}
\day{1}
\date{01.01.1970}
%\setlogo{oiglogo}%brak ustawienia wstawia domyslnie logo talentu
%\setstopka{stopka_bw}
\title{\mbox{Trójkąt}}
\id{tro}
\RAM{64}

\begin{document}

\begin{tasktext}%
    \noindent
    Dany jest trójkąt prostokątny, którego długości przyprostokątnych wynoszą odpowiednio $a=20 cm$~i~$b=30 cm$. W~narożach trójkąta umieszczono ładunki elektyczne $q_1$, $q_2$ i~$q_3$. Przyjmując, że~energia potencjalna całego układu wynosi $E_p$, oblicz wartość ładunku $q_3$ wiedząc, że~ładunki $q_1$ i~$q_2$ znajdują~się na~końcach przyprostokątnej o~długości~$a$ i przenikalność elektryczna próżni $\epsilon_0=8.85\cdot10^{12} \frac{F}{m}$.
	
    	\section{Wejście}
	W~pierwszym wierszu standardowego wejścia zapisano trzy liczby całkowite $q_1$, $q_2$ i $E_p$ $(-10^6 \leqslant q_1, q_2 \leqslant 10^6, -10^7 \leqslant E_p \leqslant 10^7)$. Dwie pierwsze są~podane w~$10^{-15} C$, natomiast energia podana jest w~$10^{-16} J$. 

	\section{Wyjście}
	Na~standardowe wyjście należy wypisać wartość ładunku $q_3$ w~pikokulombach~(przedrostek piko oznacza $10^{-12}$) z~dokładknością~do~$0.01$.
	\oigprzyklady
\end{tasktext}
\end{document}
